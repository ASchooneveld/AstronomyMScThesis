%TC:group tabular 1 1
%TC:group table 0 1
% ******************************************************************************
% ****************************** Custom Margin *********************************


% Add `custommargin' in the document class options to use this section
% Set {innerside margin / outerside margin / topmargin / bottom margin}  and
% other page dimensions
\ifsetCustomMargin
  %\RequirePackage[left=37mm,right=30mm,top=35mm,bottom=30mm]{geometry}
  \RequirePackage[left=45mm,right=40mm,top=45mm,bottom=50mm]{geometry}
  %\RequirePackage[left=40mm,right=40mm,top=45mm,bottom=45mm]{geometry}
  \setFancyHdr % To apply fancy header after geometry package is loaded
  \addtolength{\skip\footins}{1em}
\fi


\usepackage{setspace}
\usepackage{chngpage}



% Add spaces between paragraphs
%\setlength{\parskip}{0.5em}
% Ragged bottom avoids extra whitespaces between paragraphs
\raggedbottom
% To remove the excess top spacing for enumeration, list and description
%\usepackage{enumitem}
%\setlist[enumerate,itemize,description]{topsep=0em}

% *****************************************************************************
% ******************* Fonts (like different typewriter fonts etc.)*************

% Add `customfont' in the document class option to use this section

\usepackage{fontspec}
\usepackage{unicode-math,amsmath}



\ifsetCustomFont
  % Set your custom font here and use `customfont' in options. Leave empty to
  % load computer modern font (default LaTeX font).
  %\RequirePackage{helvet}

  % For use with XeLaTeX
  %  \setmainfont[
  %    Path              = ./libertine/opentype/,
  %    Extension         = .otf,
  %    UprightFont = LinLibertine_R,
  %    BoldFont = LinLibertine_RZ, % Linux Libertine O Regular Semibold
  %    ItalicFont = LinLibertine_RI,
  %    BoldItalicFont = LinLibertine_RZI, % Linux Libertine O Regular Semibold Italic
  %  ]
  %  {libertine}
%     \setmainfont[Path = ./Fonts_and_stuff/,  
%     Extension = .otf,
%     UprightFont = *Serif-Regular,
%     BoldFont = *Serif-Bold,
%     ItalicFont = *Serif-Italic,
%     BoldItalicFont = *Serif-BoldItalic]
% {Libertinus}
%     %\setmainfont{Latin Modern Roman}
%     \setmathfont{Libertinus Math}
%     \setmathfont[range = {up, it, bb, bbit, scr, cal, bfcal, frak, tt, sfup, sfit, bfup, bfit, bfscr, bffrak, bfsfup, bfsfit}, Path = ./Fonts_and_stuff/, Extension = .otf]{LibertinusMath-Regular}
    
%     \setsansfont[Path = ./Fonts_and_stuff/,
%     Extension = .otf,
%     UprightFont=*Sans-Regular,
%     BoldFont = *Sans-Bold,
%     ItalicFont = *Sans-Italic]{Libertinus}
    
    
    \setmonofont[FakeBold=0.10,BoldFont={* Light 10 Bold},BoldFeatures={FakeBold=1}]{Latin Modern Mono}
   
    
    
    
  %  % load font from system font
  %  \newfontfamily\libertinesystemfont{Linux Libertine O}
  %\newcommand{\computernmodernfont}{\fontfamily{cmr}\selectfont}
  \newcommand{\libertinusfont}{\fontfamily{Libertinus}\selectfont}
  
  %lmr for latin modern
  
\fi

% *****************************************************************************
% **************************** Custom Packages ********************************

% ************************* Algorithms and Pseudocode **************************

%\usepackage{algpseudocode}


% ********************Captions and Hyperreferencing / URL **********************

% Captions: This makes captions of figures use a boldfaced small font.
\RequirePackage[bf]{caption}

\RequirePackage[labelsep=space,tableposition=top,font=small]{caption}
%\renewcommand{\figurename}{Fig.} %to support older versions of captions.sty

\usepackage{hyperref}
\hypersetup{
    colorlinks,
    citecolor=black,
    filecolor=black,
    linkcolor=black,
    urlcolor=black
}

% *************************** Graphics and figures *****************************

\usepackage{pdflscape}
\usepackage{afterpage}

\usepackage[above]{placeins}

\let\Oldsection\section
\renewcommand{\section}{\FloatBarrier\Oldsection}

%\let\Oldsubsection\subsection
%\renewcommand{\subsection}{\FloatBarrier\Oldsubsection}

%\let\Oldsubsubsection\subsubsection
%\renewcommand{\subsubsection}{\FloatBarrier\Oldsubsubsection}

%\usepackage{rotating}
%\usepackage{wrapfig}

% Uncomment the following two lines to force Latex to place the figure.
% Use [H] when including graphics. Note 'H' instead of 'h'
\usepackage{float}
%\restylefloat{figure}

% Subcaption package is also available in the sty folder you can use that by
% uncommenting the following line
% This is for people stuck with older versions of texlive
%\usepackage{sty/caption/subcaption}
%\usepackage{subcaption}
\usepackage{subfig}

% ********************************** Tables ************************************
\usepackage{booktabs} % For professional looking tables
\usepackage{multirow}
\usepackage{arydshln}

%\usepackage{threeparttable}
\usepackage[para,online,flushleft]{threeparttablex}


%\usepackage{multicol}
\usepackage{longtable}
%\usepackage{tabularx}


% *********************************** SI Units *********************************
\usepackage{siunitx} % use this package module for SI units
\sisetup{number-unit-product = \hspace{0.16667em plus 0.08334em}, range-phrase = \text{--}, range-units = single, product-units = single}
\DeclareSIUnit\arcsec{arcsec}
\DeclareSIUnit\sqdeg{deg^{2}}
\DeclareSIUnit\magab{\text{mag AB}}
\DeclareSIUnit\mag{mag}
\DeclareSIUnit\magvega{\text{mag Vega}}
\DeclareSIUnit\sqarcmin{arcmin^{2}}
\DeclareSIUnit\arcmin{arcmin}
\DeclareSIUnit\MB{MB}
\DeclareSIUnit\GB{GB}
\DeclareSIUnit\PB{PB}
\DeclareSIUnit\TB{TB}





\usepackage{letltxmacro}

% ******************************* Line Spacing *********************************

% Choose linespacing as appropriate. Default is one-half line spacing as per the
% University guidelines

% \doublespacing
% \onehalfspacing
% \singlespacing


% ************************ Formatting / Footnote *******************************
\titleformat{\paragraph}[hang]{\normalfont\normalsize\bfseries}{\theparagraph}{1em}{}
\titlespacing*{\paragraph}{0pt}{3.25ex plus 1ex minus .2ex}{0.5em}

%DES+VIDEO break
\newcommand{\DESVIDEO}{DES+\allowbreak VIDEO }

%break words nicely according to British rules 
\usepackage[british]{babel}

\setlength\emergencystretch{0.5em}

\everymath{\medmuskip=3mu plus 1mu minus 3mu \relax}
\everymath{\thickmuskip=3mu plus 1mu minus 2mu \relax}



% Don't break enumeration (etc.) across pages in an ugly manner (default 10000)
%\clubpenalty=500
%\widowpenalty=500

%\usepackage[perpage]{footmisc} %Range of footnote options


\usepackage{listings}
\lstset{%
basicstyle=\small\ttfamily,
breaklines=true,
columns=fullflexible
}


\usepackage{enumerate}
\usepackage{enumitem}

% *****************************************************************************
% *************************** Bibliography  and References ********************
\citestyle{aa}
\usepackage{sty/jour-abrv}

%\usepackage{cleveref} %Referencing without need to explicitly state fig /table

% Add `custombib' in the document class option to use this section
\ifuseCustomBib
   \RequirePackage[square, sort, numbers, authoryear]{natbib} % CustomBib

% If you would like to use biblatex for your reference management, as opposed to the default `natbibpackage` pass the option `custombib` in the document class. Comment out the previous line to make sure you don't load the natbib package. Uncomment the following lines and specify the location of references.bib file

%\RequirePackage[backend=biber, style=numeric-comp, citestyle=numeric, sorting=nty, natbib=true]{biblatex}
%\bibliography{References/references} %Location of references.bib only for biblatex

\fi

% changes the default name `Bibliography` -> `References'
\renewcommand{\bibname}{References}


% ******************************************************************************
% ************************* User Defined Commands ******************************
% ******************************************************************************

% *********** To change the name of Table of Contents / LOF and LOT ************

%\renewcommand{\contentsname}{My Table of Contents}
%\renewcommand{\listfigurename}{My List of Figures}
%\renewcommand{\listtablename}{My List of Tables}


% ********************** TOC depth and numbering depth *************************


\setcounter{secnumdepth}{3}
\setcounter{tocdepth}{3}


% ******************************* Nomenclature *********************************

% To change the name of the Nomenclature section, uncomment the following line

%\renewcommand{\nomname}{Symbols}


% ********************************* Appendix ***********************************

% The default value of both \appendixtocname and \appendixpagename is `Appendices'. These names can all be changed via:

%\renewcommand{\appendixtocname}{List of appendices}
%\renewcommand{\appendixname}{Appndx}

% *********************** Configure Draft Mode **********************************

% Uncomment to disable figures in `draft'
%\setkeys{Gin}{draft=true}  % set draft to false to enable figures in `draft'

% These options are active only during the draft mode
% Default text is "Draft"
%\SetDraftText{DRAFT}

% Default Watermark location is top. Location (top/bottom)
%\SetDraftWMPosition{bottom}

% Draft Version - default is v1.0
%\SetDraftVersion{v1.1}

% Draft Text grayscale value (should be between 0-black and 1-white)
% Default value is 0.75
%\SetDraftGrayScale{0.8}


% ******************************** Todo Notes **********************************
%% Uncomment the following lines to have todonotes.

%\ifsetDraft
%	\usepackage[colorinlistoftodos]{todonotes}
%	\newcommand{\mynote}[1]{\todo[author=kks32,size=\small,inline,color=green!40]{#1}}
%\else
%	\newcommand{\mynote}[1]{}
%	\newcommand{\listoftodos}{}
%\fi

% Example todo: \mynote{Hey! I have a note}
